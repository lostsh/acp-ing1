\subsection{L'ACP et ses outils}
Considérons un jeu de données à $n$ variables quantitatives et $N$ individus.
Les individus seront alors représentés par des points de $\mathbb{R}^n$.
On notera $\forall i\in\llbracket1,N\rrbracket, X_i$ le ième individu de
notre jeu de données.

L'objectif de l'Analyse en Composante Principale (ACP) est de trouver une
projection permettant de diminuer au maximum la dimension de l'espace des
données $\mathbb{R}^n$ tout en limitant la quantité d'information perdue.

On notera F l'image de $\mathbb{R}^n$ par cette projection.
Afin de limiter la quantité d'information perdue, on introduit un critère
permettant de la mesurer: L'inertie $I$ qui se définit par
$$ I =
\sum_{k=1}^{N}
\left\|
X_k -
\left(
\frac1N\sum_{i=1}^N
\left(X_i\right)_j
\right)_{j\in\llbracket1,n\rrbracket}
\right\|$$

